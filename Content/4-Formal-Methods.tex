%&tex
% !TEX program = xelatex
% !TeX TS-program = xelatex
% !BIB TS-program = biber
% !TeX encoding = UTF-8
% !TeX spellcheck = en_US
% !TeX root = ../thesis.tex
%% ==============================
\chapter{Formal Specification-Based Methods}
\label{sec:formal_methods}
%% ==============================

Two of the most important aspects when programming a~\glspl{acn:plc} are validation and verification.
Validation describes the correctness of the specification compared to the intended use and verification is the correctness of the actual code compared to the specification.
Both of these aspects ensure that the system behaves as intended.
As~\glspl{acn:plc} are used for real-time applications, the importance of these factors increases even more.
The previously described methods for programming with C offered possibilities in that regard, but they did not improve significantly compared to the IEC 61131-3 languages.
Therefore in this section I am investigating the possibilities of using a formal specification to automatically program a PLC.
This allows to easier validate and verify the code that developed.
\citeauthor{Frey:2000aa}\todo{Formal methods in PLC programming} describe different approaches to verification and validation for~\acrshort{acn:plc} programming.
This paper also gives an overview of different methods that can be used to archive this.
From this overview I investigated the three most used techniques in the following section.

There a several methods for creating a formal specification of a process or controller.
One method is to use a formal language to specify the behavior in rules.
This method allows for easier verification and validation as the specification and code can checked via formal theorem solvers.
Approaches employing this method are described in subsection~\ref{sec:sub:flb}.

Another method is to use a modeling language to create a system specification.
A modeling language allows the user to create a structural and behavioral specification of the system.
This is in contrast to the formal languages that only describe behavioral details, but not structures.
They also allow for simulation of the system components making it easier to evaluate and validate the system.
In software development, model specifications like~\acrfull{acn:uml} are common practice.
The usage of a model can also provide a overview of the system and increase modularity.
In subsection~\ref{sec:sub:mb} I investigate approaches using this specification type.

The last approach I investigate are graph based specifications.
They are often used for behavioral descriptions in modeling languages but can also be used just to create a behavioral model.
They use system states and transitions between the states to model a event reaction semantic of the system.
Common approaches here are petri nets and condition graphs.
I describe these approaches in subsection~\ref{sec:sub:graph}.

\subsection{Formal Language Based}

\subsubsection{Linear Temporal Logic}

\todo[inline]{Construction and Verification of PLC Programs by LTL Specification}

\todo[inline]{Synthesizing Executable PLC Code for Robots from Scenario-Based GR(1) Specifications}

\subsubsection{PLCspecif}

\todo[inline]{PLC Code Generation Based on a Formal Specification Language}

\subsection{Model Based}

\subsubsection{UML / SysML}

\todo[inline]{Automated PLC Software Testing using adapted UML Sequence Diagrams}

\todo[inline]{A Model-Driven Approach on Object-Oriented PLC Programming for Manufacturing Systems with Regard to Usability}

\todo[inline]{Automatic Generation of Implementation in SysML-based Model-Driven Development for IEC 61131-3 Control Software}

\subsubsection{Simulink}

\todo[inline]{Model-based Verification of PLC programs using Simulink Design}

\todo[inline]{Comparison of a transformed Matlab/Simulink model into the programming language CFC on different IEC 61131-3 PLC environments}


\subsection{Graph Based}

\todo[inline]{Template-Based Generation of PLC Software from Plant Models Using Graph Representation}
