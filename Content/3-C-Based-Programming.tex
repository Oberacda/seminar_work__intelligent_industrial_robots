%&tex
% !TEX program = xelatex
% !TeX TS-program = xelatex
% !BIB TS-program = biber
% !TeX encoding = UTF-8
% !TeX spellcheck = en_US
% !TeX root = ../thesis.tex
%% ==============================
\section{C-based Programming Methods}
\label{sec:c_methods}
%% ==============================

C is one of the most used programming languages of the last decades.
Its closeness to the actual hardware, the access to low level features, like memory management, and the low overhead make it the ideal system programming language.
Today big parts of operating systems, many embedded and real-time applications are written in C.
This makes it a prime candidate for~\acrshort{acn:plc} programming.
As its initial introduction was in the year 1972~\cite{10.5555/576122}, there are many libraries and frameworks available today, that are implemented in C.
The importance as a system programming language has also sparked intensive research in the field of formal verification of C code\todo[inline,color=green]{Add citation for formal C verification programs.}.

It is a standardized ISO language~\cite{ISO:9899:2018}, with the current C18 standard.
The most common and best supported C standard is the ANSI C / C90 standard from 1990.

The low overhead of C code allows the programmer to write programs that have a fairly deterministic runtime behavior.
This makes it one of the prime candidates for real-time systems.
On the other hand, its low level of abstraction makes writing more complex code, e.g. for visualization, complex and error prone that other high-level languages.
But in order to efficiently and correctly program in C, a developer requires at least some experience in programming and concepts like pointer and memory management.
For this reason, C wasn't considered suitable for programming PLC's when they were initially developed.
Many automation engineers had a background electrical engineering rather than in computer science.
As they programmed~\acrshort{acn:plc}s, the languages abstracted the problem to be closer to electrical and logical circuits.

Nowadays many people are proficient in programming, allowing the usage of C in automation tasks.
Therefore, there are some~\acrshort{acn:plc} environments that allow for the developer to use C and C++ to program their~\acrshort{acn:plc}.
Many model driven systems and simulators allow for the generation of C code from a model; therefore, C support is an important part for automated programming when looking at non-PLC specific programming methods.
Even without being fully automated programming of a PLC itself, the extended possibility when using C / C++ as a programming language are worth investigating.
Therefore, in this section I am going describe two major PLC environments that allow for pure C/C++ programming on~\acrshort{acn:plc}'s.

\subsection{B \& R Automation Studio}
B \& R Automation Studio~\cite{b-r_automation:2020} is the~\acrshort{acn:plc} development environment for solutions by the Austrian company B\&R, a subsidiary of the swiss company ABB, one of the world's leading PLC manufactures.
It can be used to program all PLC's and Industrial-PC's manufactured by B\&R.

In the current version all the IEC 61131-3 languages and ANSI C are fully supported.
Additionally, support for C++ is possible, but this doesn't support the full feature set of the language.

A developer can define a cyclic task executed with a specific deadline in C code.
For this you must define the input and output, as well as the permanent variables, which are then inserted into every cycle execution.
In the module you can use all features of the C standard library that are not dependent on the UNIX operating system (like file system operations, etc.).
This allows for the reuse and verification of control code, as it follows the common C standards.

In addition to C, C++ is also partially supported.
This allows for object-oriented development that may be required in more complex control tasks.

Another feature is the integration of multiple different languages.
As they all are managed in the same runtime environment, the results of a C program can be used in a~\acrfull{acn:ST} or~\acrfull{acn:LD} program, and the other way around.
With this a separation of concerns can be realized, allowing for faster development and reuse of control code.

This toolkit can decrease the development time and effort of programming a~\acrshort{acn:plc} significantly.

\subsection{Beckhoff TwinCAT 3}

Another interesting approach to C/C++ based PLC programming, is the Beckhoff TwinCAT~\cite{Beckhoff:2020:2} environment.
Instead of a pure PLC only runtime, they build an operating system extension for Microsoft Windows that allows to run code with PLC semantics on industrial PC's as well as real PLC's.
This combination of industrial grade PC's and PLC's allows for an integrated system where automation controllers and higher-abstraction systems, like production databases can communicate seamlessly.
Another benefit is the combination of real-time tasks, like control loops, and non-real-time tasks, like visualization, in one system runtime.

The integration of C/C++ code is done via modules.
A module is a component in the runtime, that is executed standalone and communicates over the runtime, as a middleware layer, with other modules.
Modules can be assigned different time behaviors either as real-time or non-real-time tasks.
Additionally, modules can use advanced messages and group-based communication with one another to create more complex systems.

Another interesting feature of this development environment is that it supports the direct usage of simulation models as runtime modules.
Especially models developed in the MathWorks Simulink environment can be used.
This makes it an environment with good support for all kinds of use cases and application domains.
