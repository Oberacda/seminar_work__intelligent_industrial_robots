%&tex
% !TEX program = xelatex
% !TeX TS-program = xelatex
% !BIB TS-program = biber
% !TeX encoding = UTF-8
% !TeX spellcheck = en_US
% !TeX root = ../thesis.tex
%% ==============================
\section{C-based Programming Methods}
\label{sec:c_methods}
%% ==============================

C is one of the most used programming languages of the last decades.
Its closeness to the actual hardware, the access to low level features, like memory management, and the low overhead make it the ideal system programming language.
Today big parts of operating systems, many embedded and real-time applications are written in C.
This makes it a prime candidate for~\acrshort{acn:plc} programming.
As its initial introduction was in the year 1972~\cite{10.5555/576122}, there are many libraries and frameworks available today, that are implemented in C.
The importance as a system programming language has also sparked intensive research in the field of formal verification of C code\todo{Add citation for formal C verification programs.}.

It is a standardized ISO language~\cite{ISO:9899:2018}, with the current C18 standard.
The most common and best supported C standard is the ANSI C / C90 standard from 1990.

The low overhead of C code allows the programmer to write programs that have a fairly deterministic runtime behavior.
This makes it one of the prime candidates for real-time systems.
On the other hand, its low level of abstraction makes writing more complex code, e.g for visualization, more complex and error prone that other high level languages.
But in order to efficiently and correctly program in C, a developer requires at least some experience in programming and concepts like pointer and memory management.
For this reason, C wasn't considered suitable for programming PLC's when they were initially developed.
Many automation engineers had a background electrical engineering rather than in computer science.
As PLC's were programmed by them, they developed languages that abstracted the problem to be closer to electrical and logical circuits.

Nowadays many people are proficient in programming, allowing the usage of C in automation tasks.
Therefore there are some~\acrshort{acn:plc} environments that allow for the developer to use C and C++ to program their~\acrshort{acn:plc}.
In this section I am going to the two major environments that allow for pure C/C++ programming on~\acrshort{acn:plc}.

\subsection{B \& R Automation Studio}

\todo[inline]{B \& R Automation Studio C/C++ Module Support}

\subsection{Beckhoff TwinCAT 3}

\todo[inline]{Beckhoff TwinCAT3 industrial PC C/C++ Runtime Modules}
