%&tex
% !TEX program = xelatex
% !TeX TS-program = xelatex
% !BIB TS-program = biber
% !TeX encoding = UTF-8
% !TeX spellcheck = en_US
% !TeX root = ../thesis.tex
%% ==============================
\chapter{Conclusion}
\label{sec:conclusion}
%% ==============================

This seminar work investigates multiple approaches on automated programming of~\acrlong{acn:plc}s.
First two development environments that allow programming in higher level languages than the common IEC 61131-3 languages are presented.

Secondly, approaches to program~\acrshort{acn:plc}s using a high-level formal behavior description or a model of the process are presented.
These approaches are compared on their effectiveness, efficiency, user-acceptance and real-world applicability.
All presented approaches provided good effectiveness for specifying the behavior on a higher abstraction level.
They excel at designing larger systems with complex behaviors and provide benefits compared to the traditional PLC programming methods, especially when looking at~\acrshort{acn:LD}.
With all of them allowing for formal verification of the defined behavior they are way less prone to faults introduced in the design and implementation phase of development.
When looking at efficiency, the PLCspecif, Simulink and modAT4rMS approaches allowed the developer to progress significantly faster with a lower probability of errors compared to traditional programming.
Methods, like the ones presented in~\ref{sec:sub:ltl}, on the other hand provided lower efficiency as their complex syntax makes development more time intensive.
The approaches that used pre-existing commonly used modeling notations, like UML, SysML or Simulink, benefited from a greatly improved user-acceptance.
They also required less additional training to be used in an effective manner.
Real-world usability is a more complex topic to address, as not all presented approaches aimed towards becoming a standard approach for programming~\acrshort{acn:plc}s.
Approaches like the GRAFCET,  plcSpecif and modAT4rMS based ones described development environments and implementations of the generation algorithms for real-world usage.
The~\acrshort{acn:ltl} based approaches on the other hand, aimed more towards a theoretical method.

As~\acrshort{acn:plc}s are commonly used in safety critical environments that required exact timing behavior to ensure safe operation.
Therefore, the risk associated with automated programming are addressed.
Here I consider two major factors, the risk associated with the transformation from the model to the code and the risk associated with the runtime behavior of the system.
The transformation function must ensure that the behavior of the design is correctly mapped to the code.
To ensure this, an approach must give an extensive formulation of the transformation used.
All presented approaches provided, to some extent, a proof that their transformation function is correctly mapping the specification to the code.
Some approaches, like the PLCspecif or the modAT4rMS based approaches even provided extensive specification on the transformation, increasing the trustworthiness of their approach.
When looking at the runtime behavior, the automated programming introduces a lot of risk to the system.
The custom behavior of the models, that is mapped to the PLC cycles, makes it hard to give accurate predictions on the runtime behavior of the system.
In combination with the possible size of the generated model, this introduces large problems offsetting most of the development time advantages of automated programming methods.

In summary, there are a lot of methods for high-level and automated programming of~\acrlong{acn:plc}s.
They provide considerable advantages for designing large, complex system.
The available verification and validation methods make them less prone to faults in design and implementation.
But where they excel at efficiency and effectiveness during the design, they have significant disadvantages when looking at the deployment and the runtime guarantees that can be given.
They require in depth analysis on the deployment platform, offsetting a lot of the advantages.
Still I think that automated programming for~\acrshort{acn:plc} is a promising topic of research and with improved development toolsets for the approaches, the disadvantages can be negated almost entirely.