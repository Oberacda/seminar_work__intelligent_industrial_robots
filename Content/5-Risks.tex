%&tex
% !TEX program = xelatex
% !TeX TS-program = xelatex
% !BIB TS-program = biber
% !TeX encoding = UTF-8
% !TeX spellcheck = en_US
% !TeX root = ../thesis.tex
%% ==============================
\section{Risks of Automated Programming}
\label{sec:risks}
%% ==============================

With the primary use case of~\acrshort{acn:plc} in industrial contexts, safety and reliability of the hardware, as well as the software, is of the uttermost importance.
Failures caused by a fault in the software can result in vast amounts of property damages or even injuries for human workers.
In the case of~\acrshort{acn:plc} due to them working in a safety critical environment they must operate as a real-time system to ensure the safe state of the system.
This section investigates the risks associated with the automated program generation for~\acrshort{acn:plc}.
Details on safety features and functions are omitted, as they are a design constraint.
It is assumed that the system was designed in a way that would keep it in a safe state for common errors if the implementation behaves like specified in the design.

The main concern of the section is the risk associated with the automated programming of the~\acrshort{acn:plc}.
The risk analysis is separated into the transformation function correctness, described in~\ref{sec:sub:trans}, and the runtime guarantees, described in~\ref{sec:sub:rt}. 
%% ==============================
\subsection{Transformation Function Correctness}
\label{sec:sub:trans}
%% ==============================
The transformation function is a core component of high-level language based automatic programming.
When the transformation function is not implementing the behavior of the high-level abstraction correctly in the target source code, all verification on the high-level abstraction is meaningless.
Therefore, it must be ensured that the transformation is correctly generating the source code in accordance to the model behavior.
This indicates multiple requirements for a transformation function:
\begin{itemize}
	\item \textbf{Correctness} of the transformation of a given model behavior to source code running on a target platform must be guaranteed.
	\item \textbf{Traceability} between artifacts to allow fault tracing from the code to the model and the other way around. 
\end{itemize}
When both these points are addressed formally, an assessment of the risk of the method can be made.

\paragraph{Correctness} is only briefly addressed in most of the previously presented approaches.
This makes it difficult to address the correctness of the system formally.
Commonly papers only outline the transformation function and argue on the correctness for the general code.
None of the approaches had a full proof of correctness publicly available.
The Simulink based approach by~\citeauthor{6489667}~\cite{6489667}, the plcML and~\acrshort{acn:mod4} approaches by~\citeauthor{WITSCH2015} and~\citeauthor{Obermeier:2015aa}, and the PLCspecif language~\cite{7819191, darvas2015syntax, darvas2015requirements, darvas2015formal, 10.1007/978-3-319-33693-0_32} provide extensive proofs of execution semantic equivalence between the generated code and the model.
Therefore, these approaches have no inert risk associated with the code generation, expect for implementation faults in actual implementations of the code generator.
This makes them a more reasonable choice for real-world usage.
The other approaches argue about the correctness without going into details.
It can be assumed that in a real-world comparison these would also have an insignificant risk of code generation errors, as the general transformation rules are sound.
But without a detailed and in-depth proof the is no certainty.
An exception is the approach using GRAFCET as modeling language, as the language was designed with the execution semantics of the IEC 61131-3 implementation languages in mind.
Here the importance is in the implementation of the model in the language.
This is addressed in the paper but only the general transformation scheme is given.

\paragraph{Traceability} is addressed commonly in the previously mentioned approaches.
All presented approaches provide traceability from the model to the code.
This is either done by specific annotations in the source code, or by returning a report.
In some cases, like the Simulink to~\acrshort{acn:CFC} transformation, traceability is possible from a syntactical similarity.
Traceability from the source code to the model is a more complex problem that only few approaches support.
Reason for this, is the complexity of the implementation language compared to the modeling language.
It requires a transformation from the source code to the formalization language.
For arbitrary code in the implementation language this poses a significant challenge as it corresponds to a full source to source translation.
Often this is not possible, as the model language does not support all levels of detail from the implementation language.
Therefore, backwards transformations in the presented approaches are possible for the graphical implementation languages or if the textual code is restricted to a certain structure.
In summary, traceability between the model and source code is an important aspect to allow for risk minimization.
It can increase the readability and reduce the time it takes to find faults, therefore effectively minimizing risks associated with the automated programming.

%% ==============================
\subsection{Runtime guarantees}
\label{sec:sub:rt}
%% ==============================
Runtime guarantees are especially interesting when looking at the dynamic behavior of the system during actual operation.
In real-time systems, it is required that the system follows a fixed behavior respecting defined deadlines.
In~\enquote{traditional}~\acrshort{acn:plc} programming this is done by specifying the deadline for a given~\acrfull{acn:pou}.
This deadline is then verified either via simulation or by statically checking if the operations in the~\acrshort{acn:pou} can be executed before the deadline.
How the deadline behavior is verified is dependent on the~\acrshort{acn:plc} environment.
For example, in Siemens STEP7 there are methods for simulating the timing behavior of code on specific hardware setups.
Additionally, it provides monitoring options if the code runs on physical hardware.
Therefore, the runtime behavior of an automated programming approach must be studied on a case by case bases with the exact setup that will be used during operation.
This dependence on hardware components reduces the flexibility of the designed system but is required to make accurate guarantees on the timing behavior.
For regular control programs this can lead to redesigns, if during testing it becomes apparent that the current software cannot be executed in the required cycle time.
For automatically programmed~\acrshort{acn:plc}s this is an issue.
As the code is created from a formalization of a process or behavior, the system cannot be redesigned to reduce dynamic  runtime.
These reductions would have to be implemented as optimizations in the code generator.
This introduces new challenges as the generator must ensure that the optimizations do not disturb the designed behavior.
In all the presented approaches only then~\enquote{Simulink PLC Coder} (\ref{sec:sub:simulink}) supplied optimization methods for the generated code.
But in this case the code generator is a commercial product, where no insights into the actual transformation code is provided.

In addition to the obvious constraints originating from the automated code generation, there are other problems that are code generator specific.
Some of the code generators use custom state machines that work over multiple~\acrshort{acn:plc} cycles.
Therefore, even if the cycle times are compliant, the actual result can appear multiple cycles later than expected.
These are constraints that need to be taken into consideration when analyzing the runtime behavior of automatically programmed PLCs.

This shows that the high-level abstraction during design of the system introduces new challenges regarding the actual deployment of the code.
With the deterministic timing behavior of a PLC being one of its core advantages, not being able to verify it is a significant disadvantage.
Automatic programming might speed up the development process and reduces error proneness during development, but without methods of ensuring deterministic timing behavior, they might not be feasible for most~\acrshort{acn:plc} use cases.
