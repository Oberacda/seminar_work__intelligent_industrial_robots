%&tex
% !TEX program = xelatex
% !TeX TS-program = xelatex
% !BIB TS-program = biber
% !TeX encoding = UTF-8
% !TeX spellcheck = en_US
% !TeX root = ../thesis.tex
%% ==============================
\section{Risks of Automated Programming}
\label{sec:risks}
%% ==============================

With the primary use case of~\acrshort{acn:plc} in industrial contexts, safety and reliability of the hardware as well as the software is of the uttermost importance.
Failures caused by a fault in the software can result in huge amounts of property damages or even injuries for human workers.
In the case of~\acrshort{acn:plc} due to them working in a safety critical environment they have to operate as a real-time system to ensure the safe state of the system.
In this section, I investigate the risks associated with the automated program generation for~\acrshort{acn:plc}.
I do not go into detail on safety features and functions as they are a design constraint.
For this section I assume that the system was designed in a way that would keep it in a safe state for common errors as long as the software behaves like specified in the design.

The main concern of the section is the risk associated with the automated programming of the~\acrshort{acn:plc}.
I separate the risk into the transformation function correctness, described in~\ref{sec:sub:trans}, and the runtime guarantees, described in~\ref{sec:sub:rt}. 
%% ==============================
\subsection{Transformation Function Correctness}
\label{sec:sub:trans}
%% ==============================
The transformation function is a core component when looking at high-level language base automatic programming.
When the transformation function is not implementing the behavior of the high-level abstraction correctly to source code, all validation and verification on the design is meaningless.
Therefore it has to be ensured that the transformation is correctly generating the source code in accordance to the model behavior.
This indicates multiple requirements for a transformation function:
\begin{itemize}
	\item \textbf{Correctness} of the transformation of a given model behavior to source code running on a target platform has to be guaranteed.
	\item \textbf{Traceability} between artifacts to allow fault tracing from the code to the model and the other way around. 
\end{itemize}
When both these points are addressed formally, a assessment of the risk of the method can be made.

In the previously presented approaches on automated programming from formal definitions the topic of correctness is often only briefly addressed.
This makes it difficult to address the correctness of the system formally.

Traceability is addressed more often for the previously mentioned approaches.

%% ==============================
\subsection{Runtime guarantees}
\label{sec:sub:rt}
%% ==============================
Runtime guarantees are especially interesting when looking at the dynamic behavior of the system during actual operation.
In real-time systems, it is required that the system follows a fixed behavior respecting defined deadlines.
In~\enquote{traditional}~\acrshort{acn:plc} programming this is done by specifying the deadline for a given~\acrfull{acn:pou}.
This deadline is then verified either via simulation or by statically checking if the operations in the~\acrshort{acn:pou} can be executed before the deadline.

This approach is not helpful in case of most of the presented approaches.
They often use custom multi PLC cycle state-machines that make it harder to predict the actual execution of a certain event.

